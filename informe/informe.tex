\documentclass[12pt, a4paper]{article}

% --- Paquetes de Idioma y Codificación ---
\usepackage[utf8]{inputenc}
\usepackage[T1]{fontenc}
\usepackage[english]{babel} 

% --- Paquetes de Formato y Diseño ---
\usepackage{geometry}
\geometry{top=2.5cm, bottom=2.5cm, left=3cm, right=3cm}
\usepackage{parskip} % Espacio entre párrafos en lugar de sangría
\usepackage{setspace}
\onehalfspacing % Interlineado de 1.5

% --- Paquetes Matemáticos  ---
\usepackage{amsmath}
\usepackage{amssymb}
\usepackage{amsfonts}
\usepackage{amsthm}

% --- Paquetes para Gráficos y Figuras ---
\usepackage{graphicx}
\usepackage{float}
\usepackage{caption}
\usepackage{subcaption}

% --- Paquetes para Código (Solidity/C++) ---
\usepackage{listings}
\usepackage{xcolor}

% Configuración de colores para código
\definecolor{codegreen}{rgb}{0,0.6,0}
\definecolor{codegray}{rgb}{0.5,0.5,0.5}
\definecolor{codepurple}{rgb}{0.58,0,0.82}
\definecolor{backcolour}{rgb}{0.95,0.95,0.92}

\lstdefinestyle{mystyle}{
    backgroundcolor=\color{backcolour},   
    commentstyle=\color{codegreen},
    keywordstyle=\color{magenta},
    numberstyle=\tiny\color{codegray},
    stringstyle=\color{codepurple},
    basicstyle=\ttfamily\footnotesize,
    breakatwhitespace=false,         
    breaklines=true,                 
    captionpos=b,                    
    keepspaces=true,                 
    numbers=left,                    
    numbersep=5pt,                  
    showspaces=false,                
    showstringspaces=false,
    showtabs=false,                  
    tabsize=2
}
\lstset{style=mystyle}

% --- Hipervínculos ---
\usepackage{hyperref}
\hypersetup{
    colorlinks=true,
    linkcolor=black,
    filecolor=magenta,      
    urlcolor=blue,
    citecolor=blue,
}

% --- Datos del Documento ---
\title{\textbf{FHElect: Un sistema de votación usando Fully Homomorphic Encryption}}
\author{Fuentes \and Tievoli \and D'Elia \and}

\begin{document}

% --- Portada ---
\maketitle
\thispagestyle{empty} % Sin número de página en la portada
\newpage

% --- Resumen (Opcional) ---
\begin{abstract}
    \emph{
        Los procesos electorales son fundamentales para las democracias y la toma de decisiones en organizaciones privadas. 
        Para garantizar su legitimidad, es esencial asegurar propiedades como el anonimato y la integridad. 
        Este trabajo explora el uso de Fully Homomorphic Encryption (FHE) como solución a los desafíos 
        de los sistemas de votación electrónicos. 
        Se analiza como FHE permite el conteo verificable de votos sobre datos cifrados, 
        preservando la privacidad del votante y asegurando la transparencia.    
        }
\end{abstract}
\newpage

% --- Índice ---
\tableofcontents
\newpage

% --- Inicio del Contenido ---
\setcounter{page}{1}

\section{Introducción y Motivación}
% Aquí va el texto sobre el contexto actual, el costo de las elecciones ($230MM) y la dicotomía papel vs. digital.
A lo largo de los años los sistemas de votación han evolucionado progresivamente en la búsqueda de mayor transparencia y robustez. 
Sin embargo a medida que los procesos electorales aumentan su escala y requieren condiciones de seguridad cada vez más rígidas, 
su ejecución se ha tornado costosa e ineficiente en términos de tiempo.

En la actualidad, predominan dos paradigmas: 

Por un lado, definimos como voto tradicional al voto que se realiza fisicamente
mediante la introducción del mismo en un sobre para colocarlo en una urna. 

Por otro lado el concepto de voto electrónico abarca cualquier implementación digital diseñada para gestionar 
este proceso. 
A lo largo de este informe, se presentarán distitnas alternativas de voto electrónico.


\subsection{Voto Tradicional}

El paradigma del voto tradicional es el estándar predominante en las democracias modernas y en muchas votaciones donde 
el anonimato es fundamental. Esto se debe a la confianza histórica del sistema, y la sencillez con la que el elector
se asegura de que su voto es en efecto anonimo.

Sin embargo este sistema trae consigo varias desventajas:
\begin{itemize}
    \item{\textbf{Ineficiente temporalmente}: El recuento manual es un proceso lento que retrasa la oficialización de resultados.}
    \item{\textbf{Costoso}: La logística necesaria para garantizar unas elecciones seguras y transparentes puede generar un costo muy elevado. 
    Esto sucede especialmente en elecciones nacionales. Se estima que el costo de las elecciones nacionales en Argentina
    en 2025 fue de  \$ 230.000 millones de pesos[cite: 1].}
    \item{\textbf{Centralizado}: El sistema es vulnerable a errores humanos y fraudes, ya que el conteo depende
    de la autoridad electoral.}
\end{itemize}

\subsection{Voto Electrónico}

La motivación para implementar el voto electrónico nace de la necesidad de crear un sistema más ágil 
que elimine el recuento manual de votos y el uso masivo de boletas físicas. 
De esta manera los procesos electorales podrian resultar mucho más económicos gracias al desplome de los costos logísticos y 
permitiendo a su vez la obtención de resultados de manera instantánea.

Implementar sistemas de esta naturaleza puede implicar algunos inconvenientes:

\begin{itemize}
    \item{\textbf{Confianza}: A diferencia del sistema tradicional, en este caso los votantes no tienen la percepcion natural
    de que su voto va a ser contabilizado. En lugar de depositarlo en una urna deben confiar en que el sistema digital 
    con el que interactuan efectivamente va a realizar las operaciones esperadas}
    \item{\textbf{Centralización}: Si estos sistemas se implementan de manera centralizada persiste el riesgo de que 
    la autoridad electoral sea corrupta y manipule los resultados}
    \item{\textbf{Ataques}: Como el sistema de votaciones es software pueden aparecer vulnerabilidades de seguridad
    informaticas que pongan en riesgo el proceso.}
\end{itemize}

Las implementaciones de voto electrónico \textit{descentralizado} si bien resuelven el problema de la centralización,
traen otras complicaciones que se van a discutir luego. 

\section{Propiedades de un Sistema de Votación}
\subsection{Privacidad}
\subsection{Integridad}
\subsection{Verificabilidad Individual y Universal}
\subsection{Resistencia a la Coerción}

\section{Análisis Crítico de los Enfoques Actuales}

\subsection{Limitaciones del Voto Tradicional}
% Análisis sobre confianza histórica vs ineficiencia y centralización.

\subsection{El Problema del Voto Electrónico Centralizado}
% La "Caja Negra".

\subsection{Blockchain ``Naive'': Transparencia vs. Privacidad}
% Explicación de por qué un smart contract simple expone el voto y viola la privacidad.

\section{Planteamiento del Problema Criptográfico}
% El "Cliffhanger": ¿Cómo computar datos sin revelarlos?
% Introducción al problema que resuelve el Homomorfismo.

% --- Bibliografía ---
\begin{thebibliography}{9}

\bibitem{mit_paper}
\href{https://www.dci.mit.edu/projects/going-from-bad-to-worse-from-internet-voting-to-blockchain-voting }{
    \textit{Going from bad to worse: from internet voting to blockchain voting}. MIT DigitalCurrency Initiative.}

\bibitem{fhevm_whitepaper}
\href{https://github.com/zama-ai/fhevm/blob/main/fhevm-whitepaper.pdf }{Zama AI. \textit{FHEVM Whitepaper}.}

\bibitem{lanacion}
\href{https://www.lanacion.com.ar/politica/el-costo-de-la-eleccion-sera-de-por-lo-menos-de-230-mil-millones-nid18102025/}{La Nación. \textit{El costo de la elección será de por lo menos de 230 mil millones}. 18/10/2025.}


\end{thebibliography}

\end{document}